\documentclass{report}
% explicitly call these packages to avoid this issue:
% https://stackoverflow.com/questions/46080853/why-does-rendering-a-pdf-from-rmarkdown-require-closing-rstudio-between-renders?
% utm_medium=organic&utm_source=google_rich_qa&utm_campaign=google_rich_qa
\usepackage{booktabs}
\usepackage{longtable}
\usepackage{array}
\usepackage{multirow}
\usepackage{color}          % enables font colors
\usepackage[table]{xcolor}  % enables even more font and background colors
\usepackage{wrapfig}
\usepackage{float}
\usepackage{colortbl}
\usepackage{pdflscape}
\usepackage{tabu}
\usepackage{threeparttable}
\usepackage[normalem]{ulem}
\usepackage[export]{adjustbox}  % enables two column images at the top

% set font encoding for PDFLaTeX or XeLaTeX
\usepackage[utf8]{inputenc}   % sets document encoding to utf8
\usepackage[default]{lato}    % sets default font to Lato
\usepackage[T1]{fontenc}
\usepackage{tikz}
\usetikzlibrary{calc}
\usepackage{memhfixc}
\usepackage{amssymb}          % for itemized list styles

% document setup
\definecolor{urbnblue}{HTML}{1696D2}    % defines the Urban Institute blue color
\usepackage{enumitem}                   % bullet alignment
\setlist[2]{nosep}                      % sets the itemsep and parsep for all level two lists to 0
\setenumerate{nosep}                    % sets no itemsep for enumerate lists only
\pagenumbering{gobble}                  % disable page numbering
\usepackage[hang,flushmargin]{footmisc} % don't indent footnotes
\usepackage[document]{ragged2e}         % ragged right
\usepackage[none]{hyphenat}             % disable work breaks
\renewcommand\thefootnote{\textcolor{urbnblue}{\arabic{footnote}}} % change the color of the footnote indicator
\hypersetup{
  colorlinks,
  linkcolor=urbnblue,
  urlcolor=urbnblue,
}
\usepackage{parskip}
\setlength{\parskip}{0.1in}               % set paragraph spacing

% logo
\newcommand{\nvslogo}[0]{
  \begin{figure}
    \hspace{3in}
    \includegraphics[width=1.5in]{images/logo.jpg}
    \label{fig:logo}
  \end{figure}
}


% titles (14pt font)
\newcommand{\nvstitle}[1]{
  \begin{center}
      \textbf{\LARGE{#1}}
  \end{center}
}
  
% subtitles (12pt font)
\newcommand{\nvssubtitle}[1]{
  \begin{center}
    \Large{\textcolor{urbnblue}{#1}}
    \vspace{0.25in}    
  \end{center}
}

% authors (11pt font)
\newcommand{\nvsauthors}[1]{
  \begin{center}
    \textit{\large{#1}}
    \vspace{0.25in} 
  \end{center}
}

% heading 1 (20pt font)
\newcommand{\nvsheadingone}[1]{
  \textbf{\textcolor{urbnblue}{#1}}
}

% heading 2 (14pt font)
\newcommand{\nvsheadingtwo}[1]{
  \textbf{\textcolor{urbnblue}{#1}}
}

% figure label
\newcommand{\nvsfigurenumber}[1]{
  \textcolor{urbnblue}{\tiny{\\} \normalsize{FIGURE #1 \\}}
}

% figure title
\newcommand{\nvsfiguretitle}[1]{
  \textbf{\normalsize{#1} \\}
}

% figure source
\newcommand{\nvssource}[1]{
  \textbf{\footnotesize{#1}} \break
}

% figure note
\newcommand{\nvsnote}[1]{
  \textbf{\footnotesize{#1}} \break
}

% bullet points
\renewcommand{\labelitemi}{\color{urbnblue}\tiny$\blacksquare$}   % blue square
\renewcommand{\labelitemii}{\color{urbnblue}\tiny$\blacksquare$}  % blue square
\renewcommand{\labelitemiii}{\color{urbnblue}\tiny$\blacksquare$} % blue square
\renewcommand{\labelitemiv}{\color{urbnblue}\tiny$\blacksquare$}  % blue square

\newenvironment{nvsbullets}
{\begin{itemize}[leftmargin=*,labelindent=0.25in,labelsep=0.1875in]
\setlength{\itemsep}{0pt}
\setlength{\parskip}{0pt}
\setlength{\parsep}{0pt}}
{\end{itemize}} 

% numbered list
\newenvironment{nvsenumerate}
{\begin{enumerate}[leftmargin=*,labelindent=0.25in,labelsep=0.15625in]
\setlength{\itemsep}{0pt}
\setlength{\parskip}{0pt}
\setlength{\parsep}{0pt}}
{\end{enumerate}} 

% preamble1
\newcommand{\preambleone}[1]{
  \textbf{\normalsize{#1} \\}
}

