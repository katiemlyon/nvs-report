\documentclass[]{book}
\usepackage{lmodern}
\usepackage{amssymb,amsmath}
\usepackage{ifxetex,ifluatex}
\usepackage{fixltx2e} % provides \textsubscript
\ifnum 0\ifxetex 1\fi\ifluatex 1\fi=0 % if pdftex
  \usepackage[T1]{fontenc}
  \usepackage[utf8]{inputenc}
\else % if luatex or xelatex
  \ifxetex
    \usepackage{mathspec}
  \else
    \usepackage{fontspec}
  \fi
  \defaultfontfeatures{Ligatures=TeX,Scale=MatchLowercase}
\fi
% use upquote if available, for straight quotes in verbatim environments
\IfFileExists{upquote.sty}{\usepackage{upquote}}{}
% use microtype if available
\IfFileExists{microtype.sty}{%
\usepackage{microtype}
\UseMicrotypeSet[protrusion]{basicmath} % disable protrusion for tt fonts
}{}
\usepackage[margin=1in]{geometry}
\usepackage{hyperref}
\hypersetup{unicode=true,
            pdftitle={National Wildlife Refuge Visitor Survey: 2018 Individual Refuge Results for Dungeness National Wildlife Refuge},
            pdfauthor={Alia M. Dietsch, Colleen M. Hartel, Katie M. Lyon, Natalie R. Sexton},
            pdfborder={0 0 0},
            breaklinks=true}
\urlstyle{same}  % don't use monospace font for urls
\usepackage{longtable,booktabs}
\usepackage{graphicx,grffile}
\makeatletter
\def\maxwidth{\ifdim\Gin@nat@width>\linewidth\linewidth\else\Gin@nat@width\fi}
\def\maxheight{\ifdim\Gin@nat@height>\textheight\textheight\else\Gin@nat@height\fi}
\makeatother
% Scale images if necessary, so that they will not overflow the page
% margins by default, and it is still possible to overwrite the defaults
% using explicit options in \includegraphics[width, height, ...]{}
\setkeys{Gin}{width=\maxwidth,height=\maxheight,keepaspectratio}
\IfFileExists{parskip.sty}{%
\usepackage{parskip}
}{% else
\setlength{\parindent}{0pt}
\setlength{\parskip}{6pt plus 2pt minus 1pt}
}
\setlength{\emergencystretch}{3em}  % prevent overfull lines
\providecommand{\tightlist}{%
  \setlength{\itemsep}{0pt}\setlength{\parskip}{0pt}}
\setcounter{secnumdepth}{5}
% Redefines (sub)paragraphs to behave more like sections
\ifx\paragraph\undefined\else
\let\oldparagraph\paragraph
\renewcommand{\paragraph}[1]{\oldparagraph{#1}\mbox{}}
\fi
\ifx\subparagraph\undefined\else
\let\oldsubparagraph\subparagraph
\renewcommand{\subparagraph}[1]{\oldsubparagraph{#1}\mbox{}}
\fi

%%% Use protect on footnotes to avoid problems with footnotes in titles
\let\rmarkdownfootnote\footnote%
\def\footnote{\protect\rmarkdownfootnote}

%%% Change title format to be more compact
\usepackage{titling}

% Create subtitle command for use in maketitle
\providecommand{\subtitle}[1]{
  \posttitle{
    \begin{center}\large#1\end{center}
    }
}

\setlength{\droptitle}{-2em}

  \title{National Wildlife Refuge Visitor Survey: 2018 Individual Refuge Results
for Dungeness National Wildlife Refuge}
    \pretitle{\vspace{\droptitle}\centering\huge}
  \posttitle{\par}
    \author{Alia M. Dietsch, Colleen M. Hartel, Katie M. Lyon, Natalie R. Sexton}
    \preauthor{\centering\large\emph}
  \postauthor{\par}
    \date{}
    \predate{}\postdate{}
  
\documentclass{report}
% explicitly call these packages to avoid this issue:
% https://stackoverflow.com/questions/46080853/why-does-rendering-a-pdf-from-rmarkdown-require-closing-rstudio-between-renders?
% utm_medium=organic&utm_source=google_rich_qa&utm_campaign=google_rich_qa
\usepackage{booktabs}
\usepackage{longtable}
\usepackage{array}
\usepackage{multirow}
\usepackage{color}          % enables font colors
\usepackage[table]{xcolor}  % enables even more font and background colors
\usepackage{wrapfig}
\usepackage{float}
\usepackage{colortbl}
\usepackage{pdflscape}
\usepackage{tabu}
\usepackage{threeparttable}
\usepackage[normalem]{ulem}
\usepackage[export]{adjustbox}  % enables two column images at the top

% set font encoding for PDFLaTeX or XeLaTeX
\usepackage[utf8]{inputenc}   % sets document encoding to utf8
\usepackage[default]{lato}    % sets default font to Lato
\usepackage[T1]{fontenc}
\usepackage{tikz}
\usetikzlibrary{calc}
\usepackage{memhfixc}
\usepackage{amssymb}          % for itemized list styles

% document setup
\definecolor{urbnblue}{HTML}{1696D2}    % defines the Urban Institute blue color
\usepackage{enumitem}                   % bullet alignment
\setlist[2]{nosep}                      % sets the itemsep and parsep for all level two lists to 0
\setenumerate{nosep}                    % sets no itemsep for enumerate lists only
\pagenumbering{gobble}                  % disable page numbering
\usepackage[hang,flushmargin]{footmisc} % don't indent footnotes
\usepackage[document]{ragged2e}         % ragged right
\usepackage[none]{hyphenat}             % disable work breaks
\renewcommand\thefootnote{\textcolor{urbnblue}{\arabic{footnote}}} % change the color of the footnote indicator
\hypersetup{
  colorlinks,
  linkcolor=urbnblue,
  urlcolor=urbnblue,
}
\usepackage{parskip}
\setlength{\parskip}{0.1in}               % set paragraph spacing

% logo
\newcommand{\nvslogo}[0]{
  \begin{figure}
    \hspace{3in}
    \includegraphics[width=1.5in]{images/logo.jpg}
    \label{fig:logo}
  \end{figure}
}


% titles (14pt font)
\newcommand{\nvstitle}[1]{
  \begin{center}
      \textbf{\LARGE{#1}}
  \end{center}
}
  
% subtitles (12pt font)
\newcommand{\nvssubtitle}[1]{
  \begin{center}
    \Large{\textcolor{urbnblue}{#1}}
    \vspace{0.25in}    
  \end{center}
}

% authors (11pt font)
\newcommand{\nvsauthors}[1]{
  \begin{center}
    \textit{\large{#1}}
    \vspace{0.25in} 
  \end{center}
}

% heading 1 (20pt font)
\newcommand{\nvsheadingone}[1]{
  \textbf{\textcolor{urbnblue}{#1}}
}

% heading 2 (14pt font)
\newcommand{\nvsheadingtwo}[1]{
  \textbf{\textcolor{urbnblue}{#1}}
}

% figure label
\newcommand{\nvsfigurenumber}[1]{
  \textcolor{urbnblue}{\tiny{\\} \normalsize{FIGURE #1 \\}}
}

% figure title
\newcommand{\nvsfiguretitle}[1]{
  \textbf{\normalsize{#1} \\}
}

% figure source
\newcommand{\nvssource}[1]{
  \textbf{\footnotesize{#1}} \break
}

% figure note
\newcommand{\nvsnote}[1]{
  \textbf{\footnotesize{#1}} \break
}

% bullet points
\renewcommand{\labelitemi}{\color{urbnblue}\tiny$\blacksquare$}   % blue square
\renewcommand{\labelitemii}{\color{urbnblue}\tiny$\blacksquare$}  % blue square
\renewcommand{\labelitemiii}{\color{urbnblue}\tiny$\blacksquare$} % blue square
\renewcommand{\labelitemiv}{\color{urbnblue}\tiny$\blacksquare$}  % blue square

\newenvironment{nvsbullets}
{\begin{itemize}[leftmargin=*,labelindent=0.25in,labelsep=0.1875in]
\setlength{\itemsep}{0pt}
\setlength{\parskip}{0pt}
\setlength{\parsep}{0pt}}
{\end{itemize}} 

% numbered list
\newenvironment{nvsenumerate}
{\begin{enumerate}[leftmargin=*,labelindent=0.25in,labelsep=0.15625in]
\setlength{\itemsep}{0pt}
\setlength{\parskip}{0pt}
\setlength{\parsep}{0pt}}
{\end{enumerate}} 

% preamble1
\newcommand{\preambleone}[1]{
  \textbf{\normalsize{#1} \\}
}

\let\BeginKnitrBlock\begin \let\EndKnitrBlock\end
\begin{document}
\maketitle

{
\setcounter{tocdepth}{1}
\tableofcontents
}
\chapter*{}\label{section}
\addcontentsline{toc}{chapter}{}

``Inspiring quote about the refuge lorem ipsum dolor sit amet,
consectetur adipiscing elit. Sed vestibulum lorem id purus laoreet, eget
luctus sem suscipit. In hac habitasse platea dictumst.''

---A visitor to Dungeness National Wildlife Refuge

\begin{center}\includegraphics{refuge-info/Dungeness National Wildlife Refuge/cover} \end{center}

\chapter*{Acknowledgements}\label{acknowledgements}
\addcontentsline{toc}{chapter}{Acknowledgements}

This study was funded by the U.S. Fish and Wildlife Service's Natural
Resource Program Center, Division of Visitor Services and
Communications, and Transportation Program. The study design and survey
instrument were developed collaboratively with representatives from U.S.
Fish and Wildlife Service and researchers from The Ohio State University
(OSU). For their support and input to the study, we would like to thank
{[}names{]}; and any staff and volunteers at Dungeness National Wildlife
Refuge who assisted with the implementation of this survey effort.
Finally, we would like to especially acknowledge the following American
Conservation Experience team members for their work in implementing the
on-the-ground sampling for the 2018 survey effort: Ellen Bley, Kylie
Campbell, Michelle Ferguson, Justin Gole, James Puckett, Nicole Stagg,
and Angelica Varela. The success of this effort is largely a result of
their dedication to the project, as well as to the people who come to
explore these unique lands.

\textbf{Suggested citation:}

Dietsch, A. M., Hartel, C. M., Lyon, K. M., and Sexton, N.R. (2019).
``National Wildlife Refuge Visitor Survey: 2018 Individual Refuge
Results for Dungeness National Wildlife Refuge. The Ohio State
University, Columbus, OH.

\textbackslash{}begin\{figure\}

\{\centering \includegraphics[width=42in,keepaspectratio]{refuge-info/Dungeness National Wildlife Refuge/image1}

\}

\textbackslash{}caption\{\texttt{r\ paste(\textquotesingle{}refuge-info\textquotesingle{},\ params\$RefugeName,\ \textquotesingle{}image1caption.txt\textquotesingle{},\ sep\ =\ \textquotesingle{}/\textquotesingle{})}\}\label{fig:ref-image1}
\textbackslash{}end\{figure\}

\chapter*{Executive Summary}\label{executive-summary}
\addcontentsline{toc}{chapter}{Executive Summary}

This report describes the results of a visitor survey at Dungeness
National Wildlife Refuge (this refuge). Visitors were contacted during
two sampling periods outlined in Appendix A. A total of 291 visitor
groups were contacted to participate in the survey. Of those groups, 291
agreed to participate in the study. Questionnaires were completed and
returned by 291 visitors, resulting in a completion rate of 1\% among
those visitor groups that agreed to participate in the study and an
overall response rate of 1\% for the study.

\textbf{Key Findings:}

\begin{itemize}
\tightlist
\item
  X are the most popular activities. They are also the most commonly
  mentioned primary reason for visiting.
\item
  11\% respondents had visited the refuge before
\item
  Visitors average \texttt{r} visits per year
\item
  Visit length averages \texttt{r} hours
\item
  \textbf{The most frequently used information resource is personal
  knowledge from previous visits (86\%).} Other frequently used
  information sources are Refuge employees or volunteers (72\%) and
  Kiosks/displays/exhibits at this refuge (68\%). More than \texttt{r}
  (\texttt{r}\%) did not obtain any outside information prior to
  visiting.
\item
  Visitors opinions about this refuge
\item
  Transportation at this refuge
\item
  Visitor spending in the local communities
\item
  Enhancing future refuge visits
\end{itemize}

\chapter{Introduction}\label{intro}

The National Wildlife Refuge System plays a unique role in connecting
current and future generations to our nation's rich natural heritage. A
national wildlife refuge visit can instill a lasting passion for
wildlife and wild lands. A carefully designed set of amenities,
services, and recreational opportunities contributes to and enhances the
visitor experience, making these important connections possible.
Opportunities for outdoor recreation draw millions of people each year
to national wildlife refuges. Many visitors enjoy hiking, paddling,
wildlife viewing or nature photography. Others take part in heritage
sports such as hunting and fishing. All these activities offer visitors
a chance to unplug from the stresses of modern life and reconnect with
their natural surroundings.

\section{The National Wildlife Refuge
System}\label{the-national-wildlife-refuge-system}

Established in 1903, and managed by the U.S. Fish and Wildlife Service
(Service), The National Wildlife Refuge System (Refuge System) is the
leading network of protected lands and waters in the world specifically
dedicated to the conservation of fish, wildlife, and their habitats.
With 55 million visits per year, the Refuge System is committed to
maintaining customer satisfaction and public engagement and strives to
be the best place for people and wildlife to thrive (U.S. Fish and
Wildlife Service and U.S. Census Bureau
\protect\hyperlink{ref-USFWS2018}{2018}).

As stated in the National Wildlife Refuge Improvement Act of 1997, the
mission of the Refuge System is ``to administer a national network of
lands and waters for the conservation, management and, where
appropriate, restoration of the fish, wildlife, and plant resources and
their habitats within the United States for the benefit of present and
future generations of Americans.'' Part of achieving this mission is the
goal ``to foster understanding and instill appreciation of the diversity
and interconnectedness of fish, wildlife, and plants, and their
habitats'' and ``to provide and enhance opportunities to participate in
compatible wildlife-dependent recreation.'' There are 567 national
wildlife refuges (refuges) and 38 wetland management districts
nationwide, including possessions and territories in the Pacific and
Caribbean, encompassing 95 million acres of land (U.S. Fish and Wildlife
Service and U.S. Census Bureau \protect\hyperlink{ref-USFWS2018}{2018}).
(QUESTION: Do we add marine national monuments and submerged lands and
waters?) There is at least one refuge in every state and territory and
within an hour's drive of most major cities. This accessibility makes it
easy for communities to enjoy their wildlife heritage. The Refuge System
attracts more than 55 million visitors annually, including 39.6 million
people who observe and photograph birds and wildlife, 9.7 million who
hunt and fish, and 2.6 million teachers and students who use refuges as
outdoor classrooms.

\section{Why a survey of refuge
visitors?}\label{why-a-survey-of-refuge-visitors}

Understanding visitor perceptions of refuges and characterizing their
experiences on refuges are critical elements of managing refuges and
meeting the goals of the Refuge System. The President's Management
Agenda identifies the National Wildlife Refuge System as a high-impact
service provider. Refuges work to maintain a high level of customer
satisfaction by operating visitor centers; designing, installing and
maintaining accessible trails; constructing viewing blinds; issuing
special use permits; contracting with private concessions, and
leveraging low recreation fees for facility improvements.

Understanding changing visitor trends requires the same commitment to
robust inventory and monitoring that the Refuge System has so diligently
applied throughout its history. The national visitor survey effort
provides managers, planners, visitor services specialists, and others
with reliable baseline data. Knowing who visits refuges and what they
do, how satisfied they are with their experience, and what their
economic contributions are to the local community help refuge staff
communicate the value of the refuge, set priorities, plan better, and
track trends over time.

The purpose of the overall effort is to better understand visitor
experiences and trip characteristics on refuges across Refuge System, to
gauge visitors levels of satisfaction with existing recreational
opportunities, and to garner feedback from visitors about their trips to
inform the design of programs and facilities. The survey results are
intended to inform performance, planning, budget, and communications
goals. Results will also inform Comprehensive Conservation Plans (CCPs),
visitor services, and transportation planning processes. Baseline
information is also fundamental to the Service's Inventory and
Monitoring Initiative, and can be used to assess changes over time (U.S.
Fish and Wildlife Service \protect\hyperlink{ref-USFWS2017}{2017}).

Systematic monitoring of national wildlife refuges with 50,000 or more
visitors per year will occur on a 5-year rotating basis (approx. 36
refuges/year). Visitors are contacted onsite, and the survey is
administered by mail (with web option) once visitors return home.

The national visitor survey is a collaborative effort between the
Service, The Ohio State University (OSU), and American Conservation
Experience (ACE).

\chapter{Refuge Description}\label{refuge-description}

Dungeness NWR is located on the Olympic Peninsula in Sequim, WA. The
refuge was established in 1915 to preserve the unique habitat of the
Dungeness Spit, the longest natural sand spit in the United States. The
gravel beaches of the spit and the surrounding tide flats, eel beds, and
bay provide habitat and nesting ground for a variety of birds and other
species. While the refuge is only 773 acres, many different migratory
species use the refuge seasonally while others call it home year round.
Shorebirds migrate to the refuge in the spring and fall to feed in the
highly productive tide flats and waterfowl spend their winters in the
calm waters of the Dungeness Bay. Perhaps the most charismatic wildlife
to call the refuge home are harbor seals who raise their pups on the
isolated tip of the spit. Young salmon and steelhead feed and grow in
the eelgrass beds surrounding the spit. In addition, the second oldest
lighthouse in the state of Washington still shines bright on the refuge.
Built in 1857, the lighthouse is a popular hiking destination. The spit
continues to grow each year as the sandy bluffs along the Washington
coast erode and are deposited along the spit.

Dungeness NWR attracts over 101,000 visitors annually (U.S. Fish and
Wildlife Service, 2018, written comm.). Visitors flock from around the
world to hike along this scenic beach. The 10 mile hike out and back to
the lighthouse is an exciting challenge for all ages, from eager Boy
Scouts to retirees. Along the beach, visitors can watch bald eagles soar
overhead and see flocks of gulls resting along the driftwood. Visiting
this refuge is often part of a trip to nearby Olympic National Park and
many visitors camp at the neighboring county park. Figure
\ref{fig:ref-map1} displays a map of Dungeness NWR. For more
information, please visit \url{https://www.fws.gov/refuge/Dungeness/}.

\textbackslash{}begin\{figure\}
\includegraphics[width=19.29in]{refuge-info/Dungeness National Wildlife Refuge/map}
\textbackslash{}caption\{Map of
\texttt{params\$RefugeName}\}\label{fig:ref-map1}
\textbackslash{}end\{figure\} TODO: FIX CAPTION AND ADD MAP URL

\begin{figure}
\centering
\includegraphics{nvs-report_files/figure-latex/ref-map2-1.pdf}
\caption{\label{fig:ref-map2}Map of \texttt{params\$RefugeName}}
\end{figure}

\chapter{Methods}\label{methods}

\section{Contacting Visitors}\label{contacting-visitors}

Refuge staff identified two separate 14-day sampling periods, and one or
more locations at which to sample, that best reflected the diversity of
use and specific visitation patterns of each participating refuge. A
standardized sampling schedule was created for all refuges that included
eight randomly selected sampling shifts during each of the two sampling
periods. Sampling shifts were four hour time bands, stratified across AM
and PM as well as weekend and weekdays. In coordination with refuge
staff, any necessary customizations were made to the standardized
schedule to accommodate the identified sampling locations and to address
specific spatial and temporal patterns of visitation.

Twenty-five visitors (18 years of age or older) per sampling shift were
systematically selected, for a total of 375 willing participants per
refuge (or 200 per sampling period) to ensure an adequate sample of
completed surveys. When necessary, shifts were moved, added, or extended
to overcome logistical limitations (for example, weather or low
visitation at a particular site) in an effort to reach target numbers.

American Conservation Experience (ACE) interns and/or USFWS Human
Dimensions staff (survey recruiters) contacted visitors onsite following
a protocol provided by OSU that was designed to obtain a representative
sample. Instructions included contacting visitors across the entire
sampling shift (for example, every nth visitor for dense visitation, as
often as possible for sparse visitation) and contacting only one person
per group. Visitors were informed of the survey effort and asked to
participate. Willing participants provided their name, mailing address,
and preference for language (English or Spanish) and were given a small
token incentive (for example, a magnet or sticker) for their
participation. Survey recruiters were also instructed to record any
refusals and then proceed with the sampling protocol.

All visitors that agreed onsite to fill out a survey received the same
sequence of correspondence. This approach allowed for an assessment of
visitors' likelihood of completing the survey by their preferred survey
mode (Sexton, Miller, and Dietsch
\protect\hyperlink{ref-sexton2011}{2011}). Researchers at OSU sent the
following materials to all visitors agreeing to participate who had not
yet completed a survey at the time of each mailing (Dillman, Smyth, and
Christian \protect\hyperlink{ref-dillman2014}{2014}):

\begin{itemize}
\tightlist
\item
  A postcard mailed within 10 days of the initial onsite contact
  thanking visitors for agreeing to participate in the survey and
  inviting them to complete the survey online.
\item
  A packet mailed 14 days later consisting of a cover letter, survey,
  and postage paid envelope for returning a completed paper survey.
\item
  A reminder postcard mailed 14 days later.
\item
  A second packet mailed 7 days later consisting of another cover
  letter, survey, and postage paid envelope for returning a completed
  paper survey.
\end{itemize}

Each mailing included instructions for completing the survey online, so
visitors had an opportunity to complete an online survey with each
mailing. Those visitors indicating a preference for Spanish were sent
Spanish versions of all correspondence (including the survey). Online
survey data were exported and paper survey data were entered into
Microsoft Excel using a standardized survey codebook and data entry
procedure. All survey data were analyzed using Statistical Package for
the Social Sciences (SPSS, v.23) software.\footnote{Any use of trade,
  firm, or product names is for descriptive purposes only and does not
  imply endorsement by the U.S. Government}

\section{Sampling at This Refuge}\label{sampling-at-this-refuge}

A total of 291 visitors agreed to participate in the survey during the
two sampling periods at the identified locations at Dungeness National
Wildlife Refuge Table \ref{tab:sampling-dates}. In all, 291 visitors
completed the survey for a 100\% response rate, and 10\% margin of error
at the 95\% confidence level.\footnote{A margin of error of ± 5\% at a
  95\% confidence level, for example, means that, if a reported
  percentage is 55\%, then 95 out of 100 times, that sample estimate
  would fall between 50\% and 60\% if the same question was asked in the
  same way. The margin of error is calculated with an 80/20 response
  distribution, assuming that for a given dichotomous choice question,
  approximately 80\% of respondents would select one choice and 20\%
  would select the other choice (Salant and Dillman
  \protect\hyperlink{ref-salant1994}{1994}).}

Table: \label{tab:sampling-dates}

\section{Interpreting the Results}\label{interpreting-the-results}

The extent to which these results accurately represent the total
population of visitors to this refuge is dependent on the number of
visitors who completed the survey (sample size) and the ability of the
variation resulting from that sample to reflect the beliefs and
interests of different visitor user groups (Scheaffer et al.
\protect\hyperlink{ref-scheaffer1996}{2011}). The composition of the
sample is dependent on the ability of the standardized sampling protocol
for this study to account for the spatial and temporal patterns of
visitor use unique to each refuge. Spatially, the geographical layout
and public-use infrastructure varies widely across refuges. Some refuges
can be accessed only through a single entrance, while others have
multiple unmonitored access points across large expanses of land and
water. As a result, the degree to which sampling locations effectively
captured spatial patterns of visitor use will vary from refuge to
refuge. Temporally, the two 14-day sampling periods may not have
effectively captured all of the predominant visitor uses/activities on
some refuges during the course of a year, which may result in certain
survey measures such as visitors' self-reported ``primary activity
during their visit'' reflecting a seasonality bias. Results contained
within this report may not apply to visitors during all times of the
year or to visitors who did not visit the survey locations.

In this report, visitors who responded to the survey are referred to
simply as ``visitors.'' However, when interpreting the results for
Dungeness National Wildlife Refuge, any potential spatial and temporal
sampling limitation specific to this refuge needs to be considered when
generalizing the results to the total population of visitors. For
example, a refuge that sampled during a special event (for example,
birding festival) held during the spring may have contacted a higher
percentage of visitors who traveled greater than 50 miles to get to the
refuge than the actual number of these people who would have visited
throughout the calendar year (that is, oversampling of nonlocals).
Another refuge may not have enough nonlocal visitors in the sample to
adequately represent the beliefs and opinions of that group type. If the
sample for a specific group type (for example, nonlocals, hunters,
visitors who paid a fee) is too low (n \textless{} 30), a warning is
included in the text. Finally, the term ``this visit'' is used to
reference the visit during which people were contacted to participate in
the survey.

\chapter{Visitor and Trip Characteristics}\label{visit}

Data collected as part of this survey provides a baseline from which to
understand who currently visits the refuge, and who engages in the
multitude of activities offered by the refuge. Baseline data also allow
refuge staff and others to assess any potential changes in visitor
characteristics over time, which is core to the scientific foundation of
the Service's Inventory and Monitoring Initiative (U.S. Fish and
Wildlife Service \protect\hyperlink{ref-USFWS2017}{2017}).

\section{Visitor Characteristics}\label{visitor-characteristics}

\BeginKnitrBlock{preamble1}
In particular, depictions of the demographics (e.g., age, economic
status, education, race and ethnicity) of both local and non-local
visitors can inform refuge managers about which groups of people are
directly benefitting from what the refuge currently offers. This type of
visitor data can then be compared with future visitor data to assess
changes in visitation due to various conditions (e.g., shifts in
climate, resource, staff, or infrastructure). Visitor demographics can
also be compared to the demographic composition of nearby communities
using data from U.S. Census, the American Community Survey, or through
online tools such as Social Explorer (www.socialexplorer.com) to assess
whether the nearby community composition is reflected within refuge
visitation, which is a critical component of getting to know and relate
to the community (U.S. Fish and Wildlife Service
\protect\hyperlink{ref-USFWS2014}{2014}). Such data can also determine
what new audiences from the local community - if not currently reflected
in the visitor population - could potentially be engaged through
appropriately-targeted outreach efforts (e.g., building partnerships,
enhancing stepping stones of engagement; (U.S. Fish and Wildlife Service
\protect\hyperlink{ref-USFWS2014}{2014})).
\EndKnitrBlock{preamble1}

Visitors to Dungeness National Wildlife Refuge were 51\% male (with an
average age of 56 years) and 44\% female (with an average age of 54
years). Visitors, on average, reported they had 16 years of formal
education (equivalent to College). The median level of income was
\$50,000-\$74,999. See (AppendixB) for more demographic information.

\section{Participation in Outdoor
Activities}\label{participation-in-outdoor-activities}

\BeginKnitrBlock{preamble1}
Quality recreational experiences on refuges provide opportunities for
visitors to connect with nature and the outdoors. Specifically,
wildlife-dependent recreation, such as hunting, fishing, wildlife
observation or photography, environmental education, and interpretation,
can increase visitor appreciation and knowledge of natural resources
(U.S. Fish and Wildlife Service
\protect\hyperlink{ref-USFWS2011}{2011}). The survey collected data on
recreation participation at this refuge during the past 12 months and
the primary activity of each visitor when they were contacted about the
survey. Understanding recreation participation can help to guide the
allocation of resources, including staff and infrastructure, to ensure
visitors have quality, memorable experiences. Understanding visitor uses
of the refuge can also aid in developing programs that facilitate
meaningful interactions between visitors and refuge staff. Finally, such
information can also help to pinpoint locations on the refuge where
potential interactions over refuge uses may be perceived as incompatible
by different visitor groups. Anticipating and preventing any social
conflicts over refuge use can help create a quality experience and
foster personal and emotional connections to the refuge and its
resources (U.S. Fish and Wildlife Service
\protect\hyperlink{ref-USFWS2011}{2011}).
\EndKnitrBlock{preamble1}

Refuge visitors were asked about twenty different activities they may
have participated in at this refuge during the 12 months prior to
completing the survey as well as the primary activity they participated
in during this visit. Frequencies of all twenty activities are included
in (AppendixB).

\begin{itemize}
\tightlist
\item
  Wildlife observation (74\%)
\item
  Hiking/Walking (69\%)
\item
  Bird watching (57\%)
\end{itemize}

Refuge visitors were also given the opportunity to write-in other
activities if they were not included in the list of twenty activities.
The full list of `other reported activities are included in (AppendixC).
Some of the unique activities reported at this refuge included:

\begin{itemize}
\tightlist
\item
  Wildlife observation (74\%)
\item
  Hiking/Walking (69\%)
\item
  Bird watching (57\%)
\end{itemize}

\subsection{Primary Activities}\label{primary-activities}

The primary activities reported at this refuge included:

\begin{itemize}
\tightlist
\item
  Hiking/Walking (28\%)
\item
  Bird watching (16\%)
\item
  Wildlife observation (13\%)
\end{itemize}

\section{Visiting This Refuge}\label{visiting-this-refuge}

\BeginKnitrBlock{preamble1}
Understanding trends in overall visitation to this refuge provides
refuge staff with key information about its existing audiences.
Specifically, the survey explored the number of times visitors have been
to this refuge during the past year, including during which seasons, and
the trips taken to other refuges or other public lands for the same
primary activities. In combination with other trip characteristics
(e.g., trip purpose, time spent traveling to and visiting the refuge),
such information can be used to better understand activity and site
fidelity (i.e., likelihood of repeating specific past behavior) or
substitutability (i.e., likelihood of doing another activity or visiting
somewhere else to participate in a preferred activity). Such information
regarding site fidelity in particular can also help to identify the
success of existing communications at enticing different groups of
people to visit the refuge. Information on seasonality of visits can
help refuge staff anticipate when certain recreational activities or
programming is in high demand, an important informational need for
allocating appropriate resources.
\EndKnitrBlock{preamble1}

Some of the key questions related to visitation habits to this refuge
included:

\begin{itemize}
\tightlist
\item
  Which of the following best describes your most recent visit to this
  refuge?
\item
  How many people were in your personal group, including yourself, on
  your most recent visit to this refuge?
\item
  Approximately how many hours/minutes (one-way) did you travel from
  your home to this refuge?
\end{itemize}

\subsection{Visits to refuge}\label{visits-to-refuge}

In the last 12 months, how many times have you visited this refuge
(including this visit)?

\begin{itemize}
\tightlist
\item
  35\% of visitors visited the refuge only once in the last 12 months
\item
  11\% of visitors visited the refuge two times in the last 12 months
\end{itemize}

\subsection{Visits to other national wildlife
refuges}\label{visits-to-other-national-wildlife-refuges}

In the last 12 months, how many times have you visited other national
wildlife refuges?

\begin{itemize}
\tightlist
\item
  74\% of visitors did not visit any other national wildlife refuges in
  the last 12 months
\item
  21\% of visitors visited other national wildlife refuges only once in
  the last 12 months
\item
  25\% of visitors visited other national wildlife refuges two times in
  the last 12 months
\end{itemize}

\subsection{Visits to other public
lands}\label{visits-to-other-public-lands}

In the last 12 months, how many times have you visited other public
lands (for example, national or state parks) to participate in the same
primary activity as this visit?

\begin{itemize}
\tightlist
\item
  28\% of visitors did not visit any other public lands
\item
  9\% of visitors visited other public lands only once
\item
  14\% of visitors visited other public lands two times
\end{itemize}

\chapter{Communicating Refuge Information}\label{comm}

\BeginKnitrBlock{preamble1}
Effective communication with the public is critical for managing and
enhancing visitor experiences. Additionally, the Refuge System's success
in reaching new and diverse audiences depends on its ability to keep
pace with communication trends (U.S. Fish and Wildlife Service
\protect\hyperlink{ref-USFWS2016a}{2016}\protect\hyperlink{ref-USFWS2016a}{a}).
Understanding how visitors both seek out and share information about the
refuge and its resources using various communication channels (e.g.,
printed information, friends and family, websites, social media) can
inform the refuge's approach to communicating with visitors at different
points of their trips (before, during and after). Information in this
section can be used to assess changes in what sources are used as well
as how visitors perceive the helpfulness of Service communications over
time. Refuge staff can use communication channels preferred by visitors
to inform the public about the Refuge System's mission and
accomplishments.
\EndKnitrBlock{preamble1}

\section{Learning about This Refuge}\label{learning-about-this-refuge}

The survey asked which information sources visitors used to learn about
the refuge and its resources and the degree to which each information
source was helpful. Visitors across different ages may use different
information sources to learn about this refuge and its resources. The
full list of information sources presented to visitors is contained in
(AppendixB).

\begin{itemize}
\tightlist
\item
  Print
\item
  Digital
\end{itemize}

\section{Helpfulness of Information
Sources}\label{helpfulness-of-information-sources}

Among all visitors, the following information sources were most
frequently rated as very or extremely helpful:

\begin{itemize}
\tightlist
\item
  Personal knowledge from previous visits (86\%)
\item
  Refuge employees or volunteers (72\%)
\item
  Kiosks/displays/exhibits at this refuge (68\%)
\end{itemize}

Visitors under the age of 35 most often used these information sources
to learn about the refuge:

\begin{itemize}
\tightlist
\item
  Personal knowledge from previous visits (80\%)
\item
  Web-based map (70\%)
\item
  Refuge employees or volunteers (67\%)
\end{itemize}

Visitors over the age of 35 most often used these information sources to
learn about the refuge:

\begin{itemize}
\tightlist
\item
  Personal knowledge from previous visits (87\%)
\item
  Refuge employees or volunteers (74\%)
\item
  Kiosks/displays/exhibits at this refuge (68\%)
\end{itemize}

\chapter{Visitor Opinions about this Refuge}\label{opin}

\BeginKnitrBlock{preamble1}
A baseline understanding of visitor experiences provides a framework
from which the Refuge System can monitor trends in visitor experiences
and opinions overtime. Obtaining such a baseline is integral to the
Service's Inventory and Monitoring Initiative (U.S. Fish and Wildlife
Service \protect\hyperlink{ref-USFWS2017}{2017}), as well as warranted
if the Service is to remain relevant in the face of changing
demographics and wildlife-related interests (U.S. Fish and Wildlife
Service \protect\hyperlink{ref-USFWS2014}{2014}). As part of the survey,
visitors rated both the importance of different serv, facilities, and
opportunities on the refuge, and their satisfaction with each of the
things applicable to their refuge experience. Certain services or
recreational opportunities may be more or less central to the experience
of different segments of the visitor population, and the potential for
highly varied importance ratings needs to be considered when viewing the
satisfaction results of this analysis. For example, hunters may place
more importance on hunting opportunities and amenities, while
school-group leaders may place more importance on
educational/informational displays. Thus, a specialized user group's
importance of and satisfaction with a particular attribute may differ
from the overall importance of and satisfaction with that same attribute
for the full sample of visitors summarized in this report.
\EndKnitrBlock{preamble1}

\section{Overall Satisfaction}\label{overall-satisfaction}

The vast majority of surveyed visitors were satisfied with the refuge's
job of conserving fish, wildlife, and their habitats (83\%) and the
quality of the overall experience when visiting this refuge (87\%). The
next step is to compare the importance and satisfaction ratings for
visitor services provided by refuges in order to identify how well
different services meet visitor expectations.

refSatTitle

\section{Recreational Opportunities}\label{recreational-opportunities}

\subsection{Importance of Recreational
Opportunities}\label{importance-of-recreational-opportunities}

\begin{figure}
\centering
\includegraphics{nvs-report_files/figure-latex/recImp-1.pdf}
\caption{\label{fig:recImp}How important are the recreation opportunities?}
\end{figure}

\subsection{Satisfaction with Recreational
Opportunities}\label{satisfaction-with-recreational-opportunities}

\begin{figure}
\centering
\includegraphics{nvs-report_files/figure-latex/recSat-1.pdf}
\caption{\label{fig:recSat}How satisfied are you with the recreation
opportunities?}
\end{figure}

\section{Services and Facilities}\label{services-and-facilities}

\subsection{Importance of Facilities and
Services}\label{importance-of-facilities-and-services}

servicesImpTitle
\includegraphics{nvs-report_files/figure-latex/servicesImp-1.pdf}

\subsection{Satisfaction with Facilities and
Services}\label{satisfaction-with-facilities-and-services}

servSatTitle
\includegraphics{nvs-report_files/figure-latex/servSatPlot-1.pdf}

\section{Feeling Safe and Welcome}\label{feeling-safe-and-welcome}

\BeginKnitrBlock{preamble1}
People who rarely participate in wildlife-dependent recreation or do not
regularly visit refuges may have different perspectives on the role of a
refuge and its resources in everyday life compared to people who
regularly visit a refuge or frequently participate in wildlife-dependent
recreation. Visitors, particularly those new to `being in nature', may
have a sense of danger when outdoors or general discomfort associated
with the unknown (U.S. Fish and Wildlife Service
\protect\hyperlink{ref-USFWS2014}{2014}). Additionally, some people may
perceive the refuge to lack materials or programs accessible and
relevant to their interests, or may even believe that natural places are
unwelcoming because of historical associations with the outdoors. Thus,
an understanding of how visitors rate their feelings of being safe and
welcome on the refuge can provide important baseline information to
refuge staff when identifying potential areas for improvement and
evaluating how future efforts may enhance feelings of being safe and
welcome at refuges over time.
\EndKnitrBlock{preamble1}

safeTitle
\includegraphics{nvs-report_files/figure-latex/unnamed-chunk-10-1.pdf}

\chapter{Transportation at this Refuge}\label{trans}

\BeginKnitrBlock{preamble1}
Transportation networks connect local communities to refuges and are the
bedrock of all experiences that occur on refuges. Visitors access
refuges by car, by boat, by bike, and by plane. The Service works to
ensure that the roads, trails and parking areas are welcoming and safe
for visitors of all abilities.These networks include roads, bridges,
foot pathways, entrances, and the entire suite of transportation
features critical to visitor access and mobility. While many visitors
arrive at the refuge in private vehicles, other options (including
buses, trams, watercraft, and bicycles) are increasingly becoming a part
of the visitor experience. Thus, there is a critical need to know how
visitors perceive the safety of using these different transportation
options, as well as different transportation features (e.g., bridges,
roadways, entrances/exists) as part of their experience. The survey
asked visitors to rate both the importance and current satisfaction with
existing transportation-related features, as well as methods of
transportation visitors used to get to and around the refuge during
their visit. This information can help to identify if transportation
improvements could be made that would further enhance visitor access
and, ultimately, their overall experiences (U.S. Fish and Wildlife
Service \protect\hyperlink{ref-USFWS2011}{2011}). Additionally,
transportation alternatives within the Refuge System (Krechmer et al.
\protect\hyperlink{ref-krechmer2001}{2001}; Volpe Center
\protect\hyperlink{ref-volpe2010}{2010}) could help to improve refuge
conditions (e.g., quality of air, water, and habitat); thus, the survey
asked visitors to note different transportation methods used during
their trips, and their likelihood of using alternative transportation at
refuges in the future.
\EndKnitrBlock{preamble1}

\section{Travel To This Refuge}\label{travel-to-this-refuge}

The key mode of transportation used by visitors to travel to this refuge
was Private vehicle without trailer (86\%; fig. 11). The key mode of
transportation used by visitors to travel around this refuge was Foot
(46\%).

\textbf{Figure 11.} Modes of transportation used by visitors to travel
to and around Dungeness National Wildlife Refuge during this visit
(\emph{n} = \#\#\#). See Appendix C for a listing of ``other'' modes of
transportation. Modes of transportation with a proportion smaller than
1.5\% for both items were excluded; see Appendix B for frequencies of
all items.

\section{Transportation-related
Features}\label{transportation-related-features}

A majority of visitors to Dungeness National Wildlife Refuge thought
transportation-related items were important Figure \ref{fig:trans-imp}.
They were most satisfied with x and least satisfied with y Figure
\ref{fig:trans-sat}.

\begin{figure}
\centering
\includegraphics{nvs-report_files/figure-latex/trans-imp-1.pdf}
\caption{\label{fig:trans-imp}Importance of transportation-related features
when visiting this refuge}
\end{figure}

\begin{figure}
\centering
\includegraphics{nvs-report_files/figure-latex/trans-sat-1.pdf}
\caption{\label{fig:trans-sat}Rate how satisfied you are with the way this
refuge is managing each feature}
\end{figure}

\section{Alternative Transportation
Options}\label{alternative-transportation-options}

One goal of the Transportation Program is to provide alternative modes
and means of access to FWS managed lands as a way to enhance the
visitation experience. ``Access to FWS managed lands, where compatible
with Station purpose, should be available to visitors via multiple forms
of transportation, including public transit, bicycle, and walking.
Alternative forms of transportation can help reduce visitors' carbon
footprints, which in turn may have long term positive affects for the
natural resources we manage. Planning and building to accommodate
sustainable transportation options can help to achieve the FWS mission
(U.S. Fish and Wildlife Service
\protect\hyperlink{ref-USFWS2016b}{2016}\protect\hyperlink{ref-USFWS2016b}{b}).

Of six alternative transportation options listed on the survey, a
majority of refuge visitors were likely to use the following at
Dungeness National Wildlife Refuge in the future Figure
\ref{fig:alt-trans}:

\begin{itemize}
\tightlist
\item
  Pedestrian paths for access to this refuge from the local community
  (23\%)
\item
  Bus or tram that provides a guided tour (15\%)
\item
  Bike-share program that offers bicycles for rent on or near this
  refuge (14\%)
\end{itemize}

A majority of visitors indicated they were not likely to use Public
transit systems that stops at or near this refuge.

\begin{figure}
\centering
\includegraphics{nvs-report_files/figure-latex/alt-trans-1.pdf}
\caption{\label{fig:alt-trans}How likely visitors would be to use each
transportation option at this refuge if it were available in the
future.}
\end{figure}

\chapter{Visitor Spending in the Local Communities}\label{econ}

\BeginKnitrBlock{preamble1}
Tourists tend to buy a range of goods and services while visiting an
area, such as a city or town located adjacent to a refuge. Major
expenditure categories include lodging, food, supplies, and gasoline.
Spending associated with refuge visitation can generate considerable
economic benefits for local communities. For example, more than 46.5
million visits were made to refuges in fiscal year 2011, and generated
\$2.4 billion in sales, more than 35,000 jobs, and \$792.7 million in
employment income in regional economies (Carver and Caudill
\protect\hyperlink{ref-carver2013}{2013}). Information on the amount and
types of visitor expenditures in relation to particular recreation
activities (e.g., hunting, fishing, wildlife observation) has further
illustrated the economic importance of recreation (U.S. Fish and
Wildlife Service and U.S. Census Bureau
\protect\hyperlink{ref-USFWS2018}{2018}), which are the primary reasons
for which visitors seek out refuges and related resources. Finally,
visitor expenditure information can be used to determine the potential
economic impact of proposed refuge management alternatives into the
future.
\EndKnitrBlock{preamble1}

\section{Visitor Spending in the Local
Communities}\label{visitor-spending-in-the-local-communities}

Visitors that live within the local 50-mi area of a refuge typically
have different spending patterns than those that travel from longer
distances. During the two sampling periods, 62\% of surveyed visitors to
Dungeness National Wildlife Refuge indicated that they live within the
local 50-mi area while nonlocal visitors (30\%) stayed in the local
area, on average, for NA days. (tab:expendTable) shows summary
statistics for local and nonlocal visitor expenditures in the local
communities and at the refuge, with expenditures reported on a per
person per day basis. During the two sampling periods, nonlocal visitors
(\emph{n} = 30) spent an average of \$140.17 per person per day and
local visitors spent an average of \$105.91 per person per day in the
local area. It is important to note that summary statistics based on a
small sample size (n \textless{} 30) may not provide a reliable
representation of that population. Several factors should be considered
when estimating the economic importance of refuge-visitor spending in
the local communities. These factors include the amount of time spent at
the refuge, influence of the refuge on the visitors' decision to take
this trip, and the representativeness of primary activities of the
sample of surveyed visitors compared to the general population.
Controlling for these factors is beyond the scope of the summary
statistics presented in this report.

Table: \label{tab:spend-table}

Visitors

n

Median

Mean

SD

Min

Max

2

0

NA

NaN

NA

NA

NA

Local

2140

12.50

105.91

1371.08

0

43783.00

Nonlocal

1183

68.33

140.17

705.55

0

23166.67

1 n = number of visitors who answered both locality and expenditure
questions.

{Note: }

 For each respondent, reported expenditures were divided by the number
of persons in their group that shared expenses in order to determine the
spending per person per trip. This number was then divided by the number
of days spent in the local area to determine the spending per person per
day for each respondent. For respondents who reported spending less than
one full day in the local community, trip length was set equal to one
day. These visitor spending estimates are appropriate for the sampling
periods selected by refuge staff (see {[}Table{]} for sampling period
dates and {[}Figure{]} for the primary visitor activities in which
people participated), and may not be representative of the total
population of visitors to this refuge.

Figure: \label{fig:spend-fig}
\includegraphics{nvs-report_files/figure-latex/unnamed-chunk-14-1.pdf}

\chapter{Sharing Refuge Experiences Via Social Media}\label{smed}

\BeginKnitrBlock{preamble1}
Around 70\% of Americans use social media to connect with one another,
engage with news content, share information and entertain themselves
(Smith and Anderson \protect\hyperlink{ref-smith2018}{2018}).
Understanding the role of social media can expand the reach of Service
communication efforts and direct ways to better reach the next
generation of refuge visitors, a major emphasis of the Service's
recommendations for Conserving the Future (U.S. Fish and Wildlife
Service \protect\hyperlink{ref-USFWS2011}{2011}). Building and
maintaining an official presence on social media can also enhance
two-way conversations between visitors and refuge staff, a critical
element of the Service's Communication's Strategy (U.S. Fish and
Wildlife Service
\protect\hyperlink{ref-USFWS2016a}{2016}\protect\hyperlink{ref-USFWS2016a}{a}).
Two-way communication may also help to enhance trust of the agency by
highlighting the range of visitor experiences that are welcomed and
stories of refuges experiences from the perspective of
culturally-relevant groups (U.S. Fish and Wildlife Service
\protect\hyperlink{ref-USFWS2014}{2014}). Finally, a social media
presence can further generate awareness of the refuge and its resources
among audiences that do not use, or did not otherwise learn about the
refuge through traditional advertising outlets.

In an increasingly urbanized society, social media is how a lot of
younger generations are getting introduced to outdoor spaces for the
first time.
\EndKnitrBlock{preamble1}

Visitors were asked if they used social media to share their refuge
experience with others. Potential visitors may find these shared posts
when searching for information online about the refuge, specific
locations, or particular wildlife - or simply when searching about their
friend or family member's activities; thus, social media posts can serve
as a virtual ``word of mouth'' method of increasing awareness about the
refuge to the visitor's network and beyond.

More/Less than half of visitors (\texttt{r}``XX\%''`) reported using
social media to share their refuge experience with other people.

The full list of social media outlets presented to visitors is contained
in Appendix B.

\subsection{Social Media Sharing by Age
Group}\label{social-media-sharing-by-age-group}

While Millennials have often led older Americans in their adoption and
use of technology, including social media, there has been significant
growth in tech adoption in recent years among older generations. More
than half of Baby Boomers say they use social media.

Among all visitors, the following social media outlets were used most
often to share refuge experiences:

\begin{itemize}
\tightlist
\item
  None (47\%)
\item
  Facebook (39\%)
\item
  Instagram (14\%)
\end{itemize}

Visitors under the age of 35 most often used these social media outlets
to share refuge experiences:

\begin{itemize}
\tightlist
\item
  None (47\%)
\item
  Facebook (39\%)
\item
  Instagram (14\%)
\end{itemize}

Visitors over the age of 35 most often used these social media outlets
to share refuge experiences:

\begin{itemize}
\tightlist
\item
  None (47\%)
\item
  Facebook (39\%)
\item
  Instagram (14\%)
\end{itemize}

\chapter{Enhancing Future Visits to National Wildlife
Refuges}\label{futvis}

\section{Forecasting Future Recreation Demand at This
Refuge}\label{forecasting-future-recreation-demand-at-this-refuge}

Gaining a sense of what would encourage visitors to return (or what may
prevent them from returning) would be useful information for managers,
planners, and visitor services personnel. For example, some visitors may
prefer increased infrastructure, while others may want more access to
opportunities on the refuge. However, any management action aimed at
increasing recreation participation should be considered in relation to
possible effects on different visitor groups. For example, visitors may
report that more infrastructure would lead to increased participation in
a particular primary activity (e.g., birdwatching, fishing), but that
particular increased participation could also result in others feeling
crowded, subsequently reducing their participation. Thus, results in
this section should be considered as part of a network of potential
decisions that could affect different visitor groups disproportionally
if not carefully considered in full.

\includegraphics{nvs-report_files/figure-latex/unnamed-chunk-18-1.pdf}

\section{Visitor Programs}\label{visitor-programs}

Offering programming that is equitable to various groups of visitors at
refuges, whether located in urban areas or elsewhere, will help to build
a stronger conservation constituency into the future (U.S. Fish and
Wildlife Service \protect\hyperlink{ref-USFWS2014}{2014}). Creation and
administration of different types of programs can encourage people to
continue visiting the refuge, and further build relationships and
encourage visitation among new audiences. These opportunities can focus
a variety of interests and topics ranging from skill-building, specific
youth programs, family-based or multi-generational programs, or
showcasing local or unique culture. Programs that focus on the interests
of these varied audiences in order to foster stronger connections to the
refuge and its resources may ultimately help to enhance conservation
goals.

YOUTHPROG FAMILYPROG SKILLPROG LOCALPROG ARTPROG ACCESSPROG OTHERPROG 1
No No No No No No No 2 Yes No Yes No No No No 3 No No No No No No No 4
No Yes No No No No No 5 No No No No No No No 6 Yes No Yes No No Yes No
NOPROG 1 Yes 2 No 3 Yes 4 No 5 Yes 6 No {[}1{]} 8 `data.frame': 4078
obs. of 8 variables: \$ Programs that engage youth : Factor w/ 2 levels
``No'',``Yes'': 1 2 1 1 1 2 1 2 1 1 \ldots{} \$ Programs that focus on
family/multiple-generations : Factor w/ 2 levels ``No'',``Yes'': 1 1 1 2
1 1 1 2 1 1 \ldots{} \$ Programs that teach skills to visitors : Factor
w/ 2 levels ``No'',``Yes'': 1 2 1 1 1 2 1 1 1 1 \ldots{} \$ Programs
that highlight unique local culture : Factor w/ 2 levels ``No'',``Yes'':
1 1 1 1 1 1 1 2 2 1 \ldots{} \$ Programs that focus on creative pursuits
: Factor w/ 2 levels ``No'',``Yes'': 1 1 1 1 1 1 1 2 1 1 \ldots{} \$
Programs that support people with accessibility concerns: Factor w/ 2
levels ``No'',``Yes'': 1 1 1 1 1 2 1 1 1 1 \ldots{} \$ Other (specify) :
Factor w/ 2 levels ``No'',``Yes'': 1 1 1 1 1 1 1 1 1 1 \ldots{} \$ I do
not typically participate in refuge programs : Factor w/ 2 levels
``No'',``Yes'': 2 1 2 1 2 1 2 1 1 2 \ldots{} Programs that engage youth
1 No 2 Yes 3 No 4 No 5 No 6 Yes Programs that focus on
family/multiple-generations 1 No 2 No 3 No 4 Yes 5 No 6 No Programs that
teach skills to visitors 1 No 2 Yes 3 No 4 No 5 No 6 Yes Programs that
highlight unique local culture 1 No 2 No 3 No 4 No 5 No 6 No Programs
that focus on creative pursuits 1 No 2 No 3 No 4 No 5 No 6 No Programs
that support people with accessibility concerns Other (specify) 1 No No
2 No No 3 No No 4 No No 5 No No 6 Yes No I do not typically participate
in refuge programs 1 Yes 2 No 3 Yes 4 No 5 Yes 6 No {[}1{]} 8 Item No 1
Programs that engage youth 77.90584 2 Programs that focus on
family/multiple-generations 79.45071 3 Programs that teach skills to
visitors 68.09711 4 Programs that highlight unique local culture
73.24669 5 Programs that focus on creative pursuits 83.69299 6 Programs
that support people with accessibility concerns 87.10152 7 Other
(specify) 91.85875 8 I do not typically participate in refuge programs
52.79549 Yes 1 22.094164 2 20.549289 3 31.902894 4 26.753310 5 16.307013
6 12.898480 7 8.141246 8 47.204512
\includegraphics{nvs-report_files/figure-latex/programs-1.pdf}

`data.frame': 32 obs. of 4 variables: \$ Group: Factor w/ 4 levels
``18-34 years'',..: 1 1 1 1 1 1 1 1 2 2 \ldots{} \$ Item : Factor w/ 8
levels ``Programs that engage youth'',..: 1 2 3 4 5 6 7 8 1 2 \ldots{}
\$ No : num 70.9 76.7 55.6 63.3 72.3 \ldots{} \$ Yes : num 29.1 23.3
44.4 36.7 27.7 \ldots{}
\includegraphics{nvs-report_files/figure-latex/progAge-1.pdf}
`data.frame': 8 obs. of 6 variables: \$ Item : Factor w/ 8 levels ``I do
not typically participate in refuge programs'',..: 1 8 6 3 5 4 7 2 \$
low : num 52.8 68.1 73.2 77.9 79.5 \ldots{} \$ neutral: num 0 0 0 0 0 0
0 0 \$ high : num 47.2 31.9 26.8 22.1 20.5 \ldots{} \$ mean : num 1.47
1.32 1.27 1.22 1.21 \ldots{} \$ sd : num 0.499 0.466 0.443 0.415 0.404
\ldots{} Item high 8 I do not typically participate in refuge programs
47 3 Programs that teach skills to visitors 32 4 Programs that highlight
unique local culture 27 1 Programs that engage youth 22 2 Programs that
focus on family/multiple-generations 21 5 Programs that focus on
creative pursuits 16 6 Programs that support people with accessibility
concerns 13 7 Other (specify) 8

\chapter{Conclusion}\label{concl}

These individual refuge results provide a summary of trip
characteristics and experiences of a sample of visitors to Dungeness
National Wildlife Refuge during 2018 and are intended to inform
decision-making efforts related to visitor services and transportation
at this refuge. Additionally, the results from this survey can be used
to inform planning efforts, such as a refuge's Comprehensive
Conservation Plan. With an understanding of visitors' trip and activity
characteristics, and visitor-satisfaction ratings with existing
offerings, refuge managers are able to make informed decisions about
possible modifications (whether a reduction or enhancement) to visitor
facilities, services, or recreational opportunities. This information
can help managers gauge demand for refuge opportunities and inform both
implementation and communication strategies. Similarly, an awareness of
visitors' satisfaction ratings with various refuge offerings can help
determine if potential areas of concern need to be investigated further.
As another example of the utility of these results, community relations
may be improved or bolstered through an understanding of the value of
the refuge to visitors, whether that value is attributed to an
appreciation of the refuge's offerings, enjoyment of its recreational
activities, or spending contributions of nonlocal visitors to the local
economy. Such data about visitors and their experiences, in conjunction
with an understanding of biophysical data on the refuge and its
resources, can ensure that management decisions are consistent with the
Refuge System mission while fostering a continued public interest in
these special places.

More information about the national visitor survey, including reports
for all participating refuges, are available at
go.osu.edu/refugereports.

More information regarding the Service's Human Dimensions Branch can be
found at
\url{https://www.fws.gov/Refuges/NaturalResourcePC/humanDimensions.html}.

\appendix


\chapter*{Appendix A. Project
Methods}\label{appendix-a.-project-methods}
\addcontentsline{toc}{chapter}{Appendix A. Project Methods}

\section{Selecting Participating
Refuges}\label{selecting-participating-refuges}

The national visitor survey was conducted from March 2018 to April 2019
on 37 refuges across the Refuge System (tab:refuge-list). Each refuge
was selected for participation by regional office Visitor Services
Chiefs.

\begin{longtable}[]{@{}l@{}}
\caption{\label{tab:refuge-list}}\tabularnewline
\toprule
\begin{minipage}[b]{0.05\columnwidth}\raggedright\strut
\textbf{Refuges}\strut
\end{minipage}\tabularnewline
\midrule
\endfirsthead
\toprule
\begin{minipage}[b]{0.05\columnwidth}\raggedright\strut
\textbf{Refuges}\strut
\end{minipage}\tabularnewline
\midrule
\endhead
\begin{minipage}[t]{0.05\columnwidth}\raggedright\strut
\textbf{Pacific Region (R1)}\strut
\end{minipage}\tabularnewline
\begin{minipage}[t]{0.05\columnwidth}\raggedright\strut
Billy Frank Jr. Nisqually NWR (WA)Dungeness NWR (WA)Guam NWR (GU)Hanalei
NWR (HI)Kilauea Point NWR (HI)Ridgefield NWR (WA)Steigerwald Lake NWR
(WA)Tualatin River NWR (OR)\strut
\end{minipage}\tabularnewline
\begin{minipage}[t]{0.05\columnwidth}\raggedright\strut
\textbf{Southwest Region (R2)}\strut
\end{minipage}\tabularnewline
\begin{minipage}[t]{0.05\columnwidth}\raggedright\strut
Balcones Canyonlands NWR (TX) Hagerman NWR (TX)\strut
\end{minipage}\tabularnewline
\begin{minipage}[t]{0.05\columnwidth}\raggedright\strut
\textbf{Midwest Region (R3)}\strut
\end{minipage}\tabularnewline
\begin{minipage}[t]{0.05\columnwidth}\raggedright\strut
Crab Orchard NWR (IL) Loess Bluffs NWR (MO) Ottawa NWR (OH) Sherburne
NWR (MN) Shiawassee NWR (MI)\strut
\end{minipage}\tabularnewline
\begin{minipage}[t]{0.05\columnwidth}\raggedright\strut
\textbf{Southeast Region (R4)}\strut
\end{minipage}\tabularnewline
\begin{minipage}[t]{0.05\columnwidth}\raggedright\strut
A.R.M. Loxahatchee NWR (FL) Bayou Sauvage NWR (LA) Big Branch Marsh NWR
(LA) Cache River NWR (AR) J.N. Ding Darling NWR (FL) Okefenokee NWR (GA)
Pinckney Island NWR (SC) Sam D. Hamilton Noxubee NWR (MS) Tennessee NWR
(TN)\strut
\end{minipage}\tabularnewline
\begin{minipage}[t]{0.05\columnwidth}\raggedright\strut
\textbf{Northeast Region (R5)}\strut
\end{minipage}\tabularnewline
\begin{minipage}[t]{0.05\columnwidth}\raggedright\strut
Blackwater NWR (MD) Canaan Valley NWR (WV) Great Meadows NWR (MA) John
Heinz NWR at Tinicum (PA) Ohio River Islands NWR (PA) Prime Hook NWR
(DE) Sachuest Point NWR (RI)\strut
\end{minipage}\tabularnewline
\begin{minipage}[t]{0.05\columnwidth}\raggedright\strut
\textbf{Mountain-Prairie Region (R6)}\strut
\end{minipage}\tabularnewline
\begin{minipage}[t]{0.05\columnwidth}\raggedright\strut
Kirwin NWR (KS) Rainwater Basin WMD (NE) Sullys Hill National Game
Preserve (ND)\strut
\end{minipage}\tabularnewline
\begin{minipage}[t]{0.05\columnwidth}\raggedright\strut
\textbf{Alaska Region (R7)}\strut
\end{minipage}\tabularnewline
\begin{minipage}[t]{0.05\columnwidth}\raggedright\strut
None in 2018\strut
\end{minipage}\tabularnewline
\begin{minipage}[t]{0.05\columnwidth}\raggedright\strut
\textbf{Pacific Southwest Region (R8)}\strut
\end{minipage}\tabularnewline
\begin{minipage}[t]{0.05\columnwidth}\raggedright\strut
Desert NWR (NV) San Diego NWR (CA) San Diego Bay NWR (CA)\strut
\end{minipage}\tabularnewline
\bottomrule
\end{longtable}

\section{Developing the Survey
Instrument}\label{developing-the-survey-instrument}

Researchers at OSU developed the survey in consultation with the Service
Headquarters Office, managers, planners, and visitor services
professionals. The survey was peer-reviewed by academic and government
researchers. The survey and associated methodology were approved by the
Office of Management and Budget (OMB control \#: 0596-0236; expiration
date: 11/30/2020).

\section{Contacting Visitors}\label{contacting-visitors-1}

Refuge staff identified two separate 14-day sampling periods, and one or
more locations at which to sample, that best reflected the diversity of
use and specific visitation patterns of each participating refuge. A
standardized sampling schedule was created for all refuges that included
eight randomly selected sampling shifts during each of the two sampling
periods. Sampling shifts were four (hr) time bands, stratified across AM
and PM as well as weekend and weekdays. In coordination with refuge
staff, any necessary customizations were made to the standardized
schedule to accommodate the identified sampling locations and to address
specific spatial and temporal patterns of visitation.

Twenty visitors (18 years of age or older) per sampling shift were
systematically selected, for a total of 400 willing participants per
refuge (or 200 per sampling period) to ensure an adequate sample of
completed surveys. When necessary, shifts were moved, added, or extended
to alleviate logistical limitations (for example, weather or low
visitation at a particular site) in an effort to reach target numbers.

American Conservation Experience Interns and/or USFWS Human Dimensions
staff (survey recruiters) contacted visitors onsite following a protocol
provided by OSU that was designed to obtain a representative sample.
Instructions included contacting visitors across the entire sampling
shift (for example, every nth visitor for dense visitation, as often as
possible for sparse visitation) and contacting only one person per
group. Visitors were informed of the survey effort, given a token
incentive (for example, a small magnet or temporary tattoo), and asked
to participate. Willing participants provided their name, mailing
address, and preference for language (English or Spanish). Survey
recruiters were also instructed to record any refusals and then proceed
with the sampling protocol

All visitors that agreed onsite to fill out a survey received the same
sequence of correspondence. This approach allowed for an assessment of
visitors' likelihood of completing the survey by their preferred survey
mode (Sexton, Miller, and Dietsch
\protect\hyperlink{ref-sexton2011}{2011}). Researchers at OSU sent the
following materials to all visitors agreeing to participate who had not
yet completed a survey at the time of each mailing (Dillman, Smyth, and
Christian \protect\hyperlink{ref-dillman2014}{2014}):

\begin{itemize}
\tightlist
\item
  A postcard mailed within 10 days of the initial onsite contact
  thanking visitors for agreeing to participate in the survey and
  inviting them to complete the survey online.
\item
  A packet mailed 14 days later consisting of a cover letter, survey,
  and postage paid envelope for returning a completed paper survey.
\item
  A reminder postcard mailed 14 days later.
\item
  A second packet mailed 7 days later consisting of another cover
  letter, survey, and postage paid envelope for returning a completed
  paper survey.
\end{itemize}

Each mailing included instructions for completing the survey online, so
visitors had an opportunity to complete an online survey with each
mailing. Those visitors indicating a preference for Spanish were sent
Spanish versions of all correspondence (including the survey). Online
survey data were exported and paper survey data were entered into
Microsoft Excel using a standardized survey codebook and data entry
procedure. All survey data were analyzed using \emph{Statistical Package
for the Social Sciences} (SPSS, v.23) and R software\footnote{Any use of
  trade, firm, or product names is for descriptive purposes only and
  does not imply endorsement by the U.S. Government.}.

\section{Interpreting the Results}\label{interpreting-the-results-1}

The extent to which these results accurately represent the total
population of visitors to this refuge is dependent on the number of
visitors who completed the survey (sample size) and the ability of the
variation resulting from that sample to reflect the beliefs and
interests of different visitor user groups (Scheaffer and others, 1996).
The composition of the sample is dependent on the ability of the
standardized sampling protocol for this study to account for the spatial
and temporal patterns of visitor use unique to each refuge. Spatially,
the geographical layout and public-use infrastructure varies widely
across refuges. Some refuges can be accessed only through a single
entrance, while others have multiple unmonitored access points across
large expanses of land and water. As a result, the degree to which
sampling locations effectively captured spatial patterns of visitor use
will vary from refuge to refuge. Temporally, the two 14-day sampling
periods may not have effectively captured all of the predominant visitor
uses/activities on some refuges during the course of a year, which may
result in certain survey measures such as visitors' self-reported
``primary activity during their visit'' reflecting a seasonality bias.
Results contained within this report may not apply to visitors during
all times of the year or to visitors who did not visit the survey
locations.

In this report, visitors who responded to the survey are referred to
simply as ``visitors.'' However, when interpreting the results for
Dungeness National Wildlife Refuge, any potential spatial and temporal
sampling limitation specific to this refuge needs to be considered when
generalizing the results to the total population of visitors. For
example, a refuge that sampled during a special event (for example,
birding festival) held during the spring may have contacted a higher
percentage of visitors who traveled greater than 50 miles (mi) to get to
the refuge than the actual number of these people who would have visited
throughout the calendar year (that is, oversampling of nonlocals).
Another refuge may not have enough nonlocal visitors in the sample to
adequately represent the beliefs and opinions of that group type. If the
sample for a specific group type (for example, nonlocals, hunters,
visitors who paid a fee) is too low (\emph{n} \&lt; 30), a warning is
included in the text. Finally, the term ``this visit'' is used to
reference the visit during which people were contacted to participate in
the survey.

\section{Non-Response Bias}\label{non-response-bias}

Non-response bias is the bias that results when respondents differ in
meaningful ways from nonrespondents. Non-response bias affects the
ability to generalize survey results, to some degree and in some ways,
from the sample to the study's target population (Dillman, Smyth, and
Christian \protect\hyperlink{ref-dillman2014}{2014}, Salant and Dillman
(\protect\hyperlink{ref-salant1994}{1994})). If non-respondents are
found to differ from respondents in meaningful ways, care should be
taken when interpreting survey responses, as they may overrepresent some
segments of the target population to some degree, and may
under-represent other segments of the population to some degree.

To check for non-response bias, this study used answers to three
non-response bias questions and three observable characteristics of the
contacted visitor to compare respondents with nonrespondents. The
following questions and observations were used for evaluation of
non-response bias: - What is your primary activity at the refuge today?
- Do you live within 50 miles of this refuge? - What year were you born?

In addition to the three non-response bias questions, the following
three characteristics were observed and recorded: - Gender of the person
in the group who was first contacted by the survey recruiter - Number of
adults (18 years and older) in the group - Number of children (under 18
years) in the group

Ideally, responses or observed estimates for non-response bias variables
should be collected from all respondents and non-respondents. The
collection of information from all contacted individuals provides the
best comparison of characteristics between the respondent and
non-respondent populations. More practically, a majority of responses or
observed estimates must be present to adequately characterize both the
respondent and non-respondent populations. In this study, 70\% was
identified as the minimum percentage of valid values for non-response
variables needed for both respondent and non-respondent populations in
order to adequately characterize the populations on a given non-response
variable. All non-response variables met the minimum for 70\% valid
values among respondents and/or non-respondents \ref{tab:nonresponse}.
Correspondingly, all variables were used for non-response bias analysis.

\begin{longtable}[]{@{}lll@{}}
\caption{\label{tab:nonresponse} Number and percentage of respondents and
non-respondents with valid values for nonresponse
variable}\tabularnewline
\toprule
\textbf{Variable} & \textbf{Respondents} &
\textbf{Non-Respondents}\tabularnewline
\midrule
\endfirsthead
\toprule
\textbf{Variable} & \textbf{Respondents} &
\textbf{Non-Respondents}\tabularnewline
\midrule
\endhead
Gender & x & y\tabularnewline
Number of Adults & x & y\tabularnewline
Number of Children & x & y\tabularnewline
\bottomrule
\end{longtable}

\hypertarget{refs}{}
\hypertarget{ref-carver2013}{}
Carver, Erin, and James Caudill. 2013. \emph{Banking on Nature: The
Economic Benefits to Local Communities of National Wildlife Refuge
Visitation}. Book. U.S. Fish; Wildlife Service.

\hypertarget{ref-dillman2014}{}
Dillman, Don A., Jolene D. Smyth, and Leah Melani Christian. 2014.
\emph{Internet, Phone, Mail, and Mixed-Mode Surveys: The Tailored Design
Method}. Book. 4th ed. Hoboken: Wiley.

\hypertarget{ref-krechmer2001}{}
Krechmer, Daniel, Lewis Grimm, Daniel Hodge, Diana Mendes, and Frank
Goetzke. 2001. \emph{Federal Lands Alternative Transportation Systems
Study-Volume 3-Summary of National Ats Needs}. Book. Cambridge
Systematics, Inc.,; BRW Group, Inc., prepared for Federal Highway
Administration; Federal Transit Administration in association with
National Park Service, Bureau of Land Management,; U.S. Fish; Wildlife
Service.
\url{https://www.transit.dot.gov/sites/fta.dot.gov/files/docs/3039_study.pdf}.

\hypertarget{ref-salant1994}{}
Salant, Priscilla, and Don A. Dillman. 1994. \emph{How to Conduct Your
Own Survey}. Book. New York: Wiley.

\hypertarget{ref-scheaffer1996}{}
Scheaffer, Richard L, William Mendenhall III, R Lyman Ott, and Kenneth G
Gerow. 2011. \emph{Elementary Survey Sampling}. Book. Cengage Learning.

\hypertarget{ref-sexton2011}{}
Sexton, Natalie R., Holly M. Miller, and Alia M. Dietsch. 2011.
``Appropriate Uses and Considerations for Online Surveying in Human
Dimensions Research.'' Journal Article. \emph{Human Dimensions of
Wildlife} 16 (3): 154--63.
doi:\href{https://doi.org/10.1080/10871209.2011.572142}{10.1080/10871209.2011.572142}.

\hypertarget{ref-smith2018}{}
Smith, A., and M. Anderson. 2018. ``Social Media Use in 2018.'' Report.
Pew Research Center.
\url{http://assets.pewresearch.org/wp-content/uploads/sites/14/2018/03/01105133/PI_2018.03.01_Social-Media_FINAL.pdf}.

\hypertarget{ref-USFWS2011}{}
U.S. Fish and Wildlife Service. 2011. \emph{Conserving the Future:
Wildlife Refuges and the Next Generation}. Book. Washington, DC: U.S.
Department of the Interior, U.S. Fish; Wildlife Service, National
Wildlife Refuge System.
\url{https://www.fws.gov/refuges/pdfs/FinalDocumentConservingTheFuture.pdf}.

\hypertarget{ref-USFWS2014}{}
---------. 2014. \emph{Standards of Excellence}. Book.
\url{https://www.fws.gov/urban/soe.php}.

\hypertarget{ref-USFWS2016a}{}
---------. 2016a. \emph{National Wildlife Refuge System Communications
Strategy}. Book.

\hypertarget{ref-USFWS2016b}{}
---------. 2016b. \emph{Plan 2035: The National Long Range
Transportation Plan - Moving People, Conserving Wildlife}. Book.

\hypertarget{ref-USFWS2017}{}
---------. 2017. \emph{Inventory and Monitoring Initiative}. Book. Fort
Collins, CO: U.S. Department of the Interior, U.S. Fish; Wildlife
Service.

\hypertarget{ref-USFWS2018}{}
U.S. Fish and Wildlife Service and U.S. Census Bureau. 2018. \emph{2016
National Survey of Fishing, Hunting, and Wildlife-Associated
Recreation}. Book. Washington, DC: U.S. Department of the Interior, U.S.
Fish; Wildlife Service.
\url{https://www.census.gov/content/dam/Census/library/publications/2018/demo/fhw16-nat.pdf}.

\hypertarget{ref-volpe2010}{}
Volpe Center. 2010. \emph{Transit and Trail Connections-Assessment of
Visitor Access to National Wildlife Refuges}. Book. U.S. Department of
Transportation, Volpe National Transportation Systems Center; U.S. Fish;
Wildlife Service.
\url{https://www.transit.dot.gov/sites/fta.dot.gov/files/docs/Transit_Trails_Layout_Final_123010.pdf}.


\end{document}
